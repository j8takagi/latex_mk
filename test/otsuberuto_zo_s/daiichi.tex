 
\endnote{3字下げ}第一日曜\endnote{「第一日曜」は中見出し}
 
 オツベルときたら大したもんだ。\Ruby{稲扱}{いねこき}器械の六台も\ruby{据}{す}えつけて、のんのんのんのんのんのんと、大そろしない音をたててやっている。
 十六人の\ruby{百姓}{ひゃくしょう}どもが、顔をまるっきりまっ赤にして足で\ruby{踏}{ふ}んで器械をまわし、小山のように積まれた稲を片っぱしから\ruby{扱}{こ}いて行く。\ruby{藁}{わら}はどんどんうしろの方へ投げられて、また新らしい山になる。そこらは、\ruby{籾}{もみ}や藁から\ruby{発}{た}ったこまかな\ruby{塵}{ちり}で、変にぼうっと黄いろになり、まるで\ruby{沙漠}{さばく}のけむりのようだ。
 そのうすくらい仕事場を、オツベルは、大きな\ruby{琥珀}{こはく}のパイプをくわえ、\ruby{吹殻}{ふきがら}を藁に落さないよう、\ruby{眼}{め}を細くして気をつけながら、両手を背中に組みあわせて、ぶらぶら\ruby{往}{い}ったり来たりする。
 小屋はずいぶん\ruby{頑丈}{がんじょう}で、学校ぐらいもあるのだが、何せ新式稲扱器械が、六台もそろってまわってるから、のんのんのんのんふるうのだ。中にはいるとそのために、すっかり腹が\ruby{空}{す}くほどだ。そしてじっさいオツベルは、そいつで上手に腹をへらし、ひるめしどきには、六寸ぐらいのビフテキだの、\ruby{雑巾}{ぞうきん}ほどあるオムレツの、ほくほくしたのをたべるのだ。
 とにかく、そうして、のんのんのんのんやっていた。
 そしたらそこへどういうわけか、その、白象がやって来た。白い象だぜ、ペンキを\ruby{塗}{ぬ}ったのでないぜ。どういうわけで来たかって?{}そいつは象のことだから、たぶんぶらっと森を出て、ただなにとなく来たのだろう。
 そいつが小屋の入口に、ゆっくり顔を出したとき、百姓どもはぎょっとした。なぜぎょっとした?{}よくきくねえ、何をしだすか知れないじゃないか。かかり合っては大へんだから、どいつもみな、いっしょうけんめい、じぶんの稲を扱いていた。
 ところがそのときオツベルは、ならんだ器械のうしろの方で、ポケットに手を入れながら、ちらっと\ruby{鋭}{するど}く象を見た。それからすばやく下を向き、何でもないというふうで、いままでどおり往ったり来たりしていたもんだ。
 するとこんどは白象が、\RUBY{片脚}{かたあし}\RUBY{床}{ゆか}にあげたのだ。百姓どもはぎょっとした。それでも仕事が\ruby{忙}{いそが}しいし、かかり合ってはひどいから、そっちを見ずに、やっぱり稲を扱いていた。
 オツベルは\ruby{奥}{おく}のうすくらいところで両手をポケットから出して、も一度ちらっと象を見た。それからいかにも\ruby{退屈}{たいくつ}そうに、わざと大きなあくびをして、両手を頭のうしろに組んで、行ったり来たりやっていた。ところが象が\ruby{威勢}{いせい}よく、\Ruby{前肢}{まえあし}二つつきだして、小屋にあがって来ようとする。百姓どもはぎくっとし、オツベルもすこしぎょっとして、大きな琥珀のパイプから、ふっとけむりをはきだした。それでもやっぱりしらないふうで、ゆっくりそこらをあるいていた。
 そしたらとうとう、象がのこのこ上って来た。そして器械の前のとこを、\ruby{呑気}{のんき}にあるきはじめたのだ。
 ところが何せ、器械はひどく\ruby{廻}{まわ}っていて、\ruby{籾}{もみ}は夕立か\ruby{霰}{あられ}のように、パチパチ象にあたるのだ。象はいかにもうるさいらしく、小さなその眼を細めていたが、またよく見ると、たしかに少しわらっていた。
 オツベルはやっと\ruby{覚悟}{かくご}をきめて、\Ruby{稲扱}{いねこき}器械の前に出て、象に話をしようとしたが、そのとき象が、とてもきれいな、\ruby{鶯}{うぐいす}みたいないい声で、こんな文句を\ruby{云}{い}ったのだ。
「ああ、だめだ。あんまりせわしく、砂がわたしの歯にあたる。」
 まったく籾は、パチパチパチパチ歯にあたり、またまっ白な頭や首にぶっつかる。
 さあ、オツベルは\ruby{命懸}{いのちが}けだ。パイプを右手にもち直し、度胸を据えて\ruby{斯}{こ}う云った。
「どうだい、\ruby{此処}{ここ}は\ruby{面白}{おもしろ}いかい。」
「面白いねえ。」象がからだを\ruby{斜}{なな}めにして、眼を細くして返事した。
「ずうっとこっちに居たらどうだい。」
 百姓どもははっとして、息を殺して象を見た。オツベルは云ってしまってから、にわかにがたがた\ruby{顫}{ふる}え出す。ところが象はけろりとして
「居てもいいよ。」と答えたもんだ。
「そうか。それではそうしよう。そういうことにしようじゃないか。」オツベルが顔をくしゃくしゃにして、まっ赤になって\ruby{悦}{よろこ}びながらそう云った。
 どうだ、そうしてこの象は、もうオツベルの財産だ。いまに見たまえ、オツベルは、あの白象を、はたらかせるか、サーカス団に売りとばすか、どっちにしても万円以上もうけるぜ。
 
