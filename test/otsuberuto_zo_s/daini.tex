\section*{第二日曜}
 
 オツベルときたら大したもんだ。それにこの前稲扱小屋で、うまく自分のものにした、象もじっさい大したもんだ。力も二十馬力もある。第一みかけがまっ白で、\ruby{牙}{きば}はぜんたいきれいな\ruby{象牙}{ぞうげ}でできている。皮も全体、立派で\ruby{丈夫}{じょうぶ}な象皮なのだ。そしてずいぶんはたらくもんだ。けれどもそんなに\ruby{稼}{かせ}ぐのも、やっぱり主人が\ruby{偉}{えら}いのだ。
「おい、お前は時計は\ruby{要}{い}らないか。」丸太で建てたその象小屋の前に来て、オツベルは琥珀のパイプをくわえ、顔をしかめて斯う\ruby{訊}{き}いた。
「ぼくは時計は要らないよ。」象がわらって返事した。
「まあ持って見ろ、いいもんだ。」斯う言いながらオツベルは、ブリキでこさえた大きな時計を、象の首からぶらさげた。
「なかなかいいね。」象も云う。
「\ruby{鎖}{くさり}もなくちゃだめだろう。」オツベルときたら、百キロもある鎖をさ、その前肢にくっつけた。
「うん、なかなか鎖はいいね。」三あし歩いて象がいう。
「\ruby{靴}{くつ}をはいたらどうだろう。」
「ぼくは靴などはかないよ。」
「まあはいてみろ、いいもんだ。」オツベルは顔をしかめながら、赤い張子の大きな靴を、象のうしろのかかとにはめた。
「なかなかいいね。」象も云う。
「靴に\ruby{飾}{かざ}りをつけなくちゃ。」オツベルはもう大急ぎで、四百キロある分銅を靴の上から、\ruby{穿}{は}め込んだ。
「うん、なかなかいいね。」象は二あし歩いてみて、さもうれしそうにそう云った。
 次の日、ブリキの大きな時計と、やくざな紙の靴とはやぶけ、象は鎖と分銅だけで、大よろこびであるいて\ruby{居}{お}った。
「済まないが税金も高いから、今日はすこうし、川から水を\ruby{汲}{く}んでくれ。」オツベルは両手をうしろで組んで、顔をしかめて象に云う。
「ああ、ぼく水を汲んで来よう。もう何ばいでも汲んでやるよ。」
 象は眼を細くしてよろこんで、そのひるすぎに五十だけ、川から水を汲んで来た。そして菜っ葉の畑にかけた。
 夕方象は小屋に居て、十\Ruby{把}{ぱ}の\ruby{藁}{わら}をたべながら、西の三日の月を見て、
「ああ、\ruby{稼}{かせ}ぐのは\ruby{愉快}{ゆかい}だねえ、さっぱりするねえ」と云っていた。
「済まないが税金がまたあがる。今日は少うし森から、たきぎを運んでくれ」オツベルは\ruby{房}{ふさ}のついた赤い\ruby{帽子}{ぼうし}をかぶり、両手をかくしにつっ込んで、次の日象にそう言った。
「ああ、ぼくたきぎを持って来よう。いい天気だねえ。ぼくはぜんたい森へ行くのは大すきなんだ」象はわらってこう言った。
 オツベルは少しぎょっとして、パイプを手からあぶなく落としそうにしたがもうあのときは、象がいかにも愉快なふうで、ゆっくりあるきだしたので、また安心してパイプをくわえ、小さな\ruby{咳}{せき}を一つして、百姓どもの仕事の方を見に行った。
 そのひるすぎの半日に、象は九百把たきぎを運び、眼を細くしてよろこんだ。
 晩方象は小屋に居て、八把の藁をたべながら、西の四日の月を見て
「ああ、せいせいした。サンタマリア」と\ruby{斯}{こ}うひとりごとしたそうだ。
 その次の日だ、
「済まないが、税金が五倍になった、今日は少うし\ruby{鍛冶場}{かじば}へ行って、炭火を\ruby{吹}{ふ}いてくれないか」
「ああ、吹いてやろう。本気でやったら、ぼく、もう、息で、石もなげとばせるよ」
 オツベルはまたどきっとしたが、気を落ち付けてわらっていた。
 象はのそのそ鍛冶場へ行って、べたんと肢を折って\ruby{座}{すわ}り、ふいごの代りに半日炭を吹いたのだ。
 その晩、象は象小屋で、七\Ruby{把}{わ}の藁をたべながら、空の五日の月を見て
「ああ、つかれたな、うれしいな、サンタマリア」と斯う言った。
 どうだ、そうして次の日から、象は朝からかせぐのだ。藁も昨日はただ五把だ。よくまあ、五把の藁などで、あんな力がでるもんだ。
 じっさい象はけいざいだよ。それというのもオツベルが、頭がよくてえらいためだ。オツベルときたら大したもんさ。
 
