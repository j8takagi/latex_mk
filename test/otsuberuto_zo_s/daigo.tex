\endnote{3字下げ}第五日曜\endnote{「第五日曜」は中見出し}
 
 オツベルかね、そのオツベルは、おれも云おうとしてたんだが、居なくなったよ。
 まあ落ちついてききたまえ。前にはなしたあの象を、オツベルはすこしひどくし過ぎた。しかたがだんだんひどくなったから、象がなかなか笑わなくなった。時には赤い\ruby{竜}{りゅう}の眼をして、じっとこんなにオツベルを見おろすようになってきた。
 ある晩象は象小屋で、三把の藁をたべながら、十日の月を\ruby{仰}{あお}ぎ見て、
「苦しいです。サンタマリア。」と云ったということだ。
 こいつを聞いたオツベルは、ことごと象につらくした。
 ある晩、象は象小屋で、ふらふら\ruby{倒}{たお}れて地べたに座り、藁もたべずに、十一日の月を見て、
「もう、さようなら、サンタマリア。」と斯う言った。
「おや、何だって?{}さよならだ?」月が\ruby{俄}{にわ}かに象に\ruby{訊}{き}く。
「ええ、さよならです。サンタマリア。」
「何だい、なりばかり大きくて、からっきし\ruby{意気地}{いくじ}のないやつだなあ。仲間へ手紙を書いたらいいや。」月がわらって斯う云った。
「お筆も紙もありませんよう。」象は細ういきれいな声で、しくしくしくしく泣き出した。
「そら、これでしょう。」すぐ眼の前で、\ruby{可愛}{かあい}い子どもの声がした。象が頭を上げて見ると、赤い着物の童子が立って、\ruby{硯}{すずり}と紙を\ruby{捧}{ささ}げていた。象は早速手紙を書いた。
「ぼくはずいぶん眼にあっている。みんなで出て来て助けてくれ。」
 童子はすぐに手紙をもって、林の方へあるいて行った。
 \ruby{赤衣}{せきい}の童子が、そうして山に着いたのは、ちょうどひるめしごろだった。このとき山の象どもは、\ruby{沙羅樹}{さらじゅ}の下のくらがりで、\ruby{碁}{ご}などをやっていたのだが、額をあつめてこれを見た。
「ぼくはずいぶん眼にあっている。みんなで出てきて助けてくれ。」
 象は一せいに立ちあがり、まっ黒になって\ruby{吠}{ほ}えだした。
「オツベルをやっつけよう」議長の象が高く\ruby{叫}{さけ}ぶと、
「おう、でかけよう。グララアガア、グララアガア。」みんながいちどに呼応する。
 さあ、もうみんな、\ruby{嵐}{あらし}のように林の中をなきぬけて、グララアガア、グララアガア、野原の方へとんで行く。どいつもみんなきちがいだ。小さな木などは根こぎになり、\ruby{藪}{やぶ}や何かもめちゃめちゃだ。グワア グワア グワア グワア、花火みたいに野原の中へ飛び出した。それから、何の、走って、走って、とうとう向うの青くかすんだ野原のはてに、オツベルの\ruby{邸}{やしき}の黄いろな屋根を\ruby{見附}{みつ}けると、象はいちどに\ruby{噴火}{ふんか}した。
 グララアガア、グララアガア。その時はちょうど一時半、オツベルは皮の\ruby{寝台}{しんだい}の上でひるねのさかりで、\ruby{烏}{からす}の\ruby{夢}{ゆめ}を見ていたもんだ。あまり大きな音なので、オツベルの家の百姓どもが、門から少し外へ出て、小手をかざして向うを見た。林のような象だろう。汽車より早くやってくる。さあ、まるっきり、血の気も失せてかけ\ruby{込}{こ}んで、
「\ruby{旦那}{だんな}あ、象です。押し寄せやした。旦那あ、象です。」と声をかぎりに叫んだもんだ。
 ところがオツベルはやっぱりえらい。眼をぱっちりとあいたときは、もう何もかもわかっていた。
「おい、象のやつは小屋にいるのか。居る?{}居る?{}居るのか。よし、戸をしめろ。戸をしめるんだよ。早く象小屋の戸をしめるんだ。ようし、早く丸太を持って来い。とじこめちまえ、\ruby{畜生}{ちくしょう}めじたばたしやがるな、丸太をそこへしばりつけろ。何ができるもんか。わざと力を減らしてあるんだ。ようし、もう五六本持って来い。さあ、大丈夫だ。大丈夫だとも。あわてるなったら。おい、みんな、こんどは門だ。門をしめろ。かんぬきをかえ。つっぱり。つっぱり。そうだ。おい、みんな心配するなったら。しっかりしろよ。」オツベルはもう\ruby{支度}{したく}ができて、ラッパみたいないい声で、百姓どもをはげました。ところがどうして、百姓どもは気が気じゃない。こんな主人に巻き\ruby{添}{ぞ}いなんぞ食いたくないから、みんなタオルやはんけちや、よごれたような白いようなものを、ぐるぐる\ruby{腕}{うで}に巻きつける。降参をするしるしなのだ。
 オツベルはいよいよやっきとなって、そこらあたりをかけまわる。オツベルの犬も気が立って、火のつくように\ruby{吠}{ほ}えながら、やしきの中をはせまわる。
 間もなく地面はぐらぐらとゆられ、そこらはばしゃばしゃくらくなり、象はやしきをとりまいた。グララアガア、グララアガア、その\ruby{恐}{おそ}ろしいさわぎの中から、
「今助けるから安心しろよ。」やさしい声もきこえてくる。
「ありがとう。よく来てくれて、ほんとに\ruby{僕}{ぼく}はうれしいよ。」象小屋からも声がする。さあ、そうすると、まわりの象は、一そうひどく、グララアガア、グララアガア、\ruby{塀}{へい}のまわりをぐるぐる走っているらしく、度々中から、\ruby{怒}{おこ}ってふりまわす鼻も見える。けれども塀はセメントで、中には鉄も入っているから、なかなか象もこわせない。塀の中にはオツベルが、たった一人で叫んでいる。百姓どもは眼もくらみ、そこらをうろうろするだけだ。そのうち外の象どもは、仲間のからだを台にして、いよいよ塀を\ruby{越}{こ}しかかる。だんだんにゅうと顔を出す。その\ruby{皺}{しわ}くちゃで灰いろの、大きな顔を見あげたとき、オツベルの犬は気絶した。さあ、オツベルは\ruby{射}{う}ちだした。六連発のピストルさ。ドーン、グララアガア、ドーン、グララアガア、ドーン、グララアガア、ところが\ruby{弾丸}{たま}は通らない。\ruby{牙}{きば}にあたればはねかえる。一\Ruby{疋}{ぴき}なぞは\ruby{斯}{こ}う言った。
「なかなかこいつはうるさいねえ。ぱちぱち顔へあたるんだ。」
 オツベルはいつかどこかで、こんな文句をきいたようだと思いながら、ケースを帯からつめかえた。そのうち、象の片脚が、塀からこっちへはみ出した。それからも一つはみ出した。五匹の象が一ぺんに、塀からどっと落ちて来た。オツベルはケースを握ったまま、もうくしゃくしゃに\ruby{潰}{つぶ}れていた。早くも門があいていて、グララアガア、グララアガア、象がどしどしなだれ込む。
「\ruby{牢}{ろう}はどこだ。」みんなは小屋に押し寄せる。丸太なんぞは、マッチのようにへし折られ、あの白象は大へん\ruby{瘠}{や}せて小屋を出た。
「まあ、よかったねやせたねえ。」みんなはしずかにそばにより、鎖と銅をはずしてやった。
「ああ、ありがとう。ほんとにぼくは助かったよ。」白象はさびしくわらってそう云った。
 おや〔一字不明〕、川へはいっちゃいけないったら。
 
 
 
